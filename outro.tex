\section{Заключение}

В этом эссе мы рассмотрели некоторые исторические и методологические моменты связанные с формальной верификацией программ. Мы увидели как появилось понятие качества программного обеспечения, и увидели, какими методами оно обеспечивается. Некоторые из этих методов, например сама формальная верификация, была рассмотрены чуть более подробно.

Мы увидели её теоретические основы, какими методами она оперирует и отметили ограничения, которые присущи вышерассмотренным методам. Следует понимать, что рассмотренные методы являются далеко не единственными. Формальная верификация использует множество других методов, которые мы оставим за гранью этой работы.

В заключение отметим, что несмотря на большую выразительную силу, формальная верификация программ может быть избыточна в некоторых областях разработки ПО. Это, в первую очередь, связано с тем, что внедрения этих методов может потребоваться существенно большее число усилий, чем необходимо собственно для разработки, а цена ошибки в программе может быть невелика. В этих случаях обычно используют другие, менее затратные методы обеспечения качества ПО. 
