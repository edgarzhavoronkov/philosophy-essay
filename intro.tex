\section{Введение}

Вычислительная техника и программное обеспечение\footnote{Я буду в равной степени употреблять термин "программное обеспечение" и соответствующую аббревиатуру -- ПО} уже настолько прочно вошли в обиход человечества, что невозможно представить нашу жизнь без них. Программное обеспечение решает задачи автоматизации процессов в самых разных областях, начиная от простейших вычислений и заканчивая медициной или атомной энергетикой. В зависимости от задачи, возложенной на программное обеспечение, отличаются и предъявляемые к нему требования. В том числе, требования к надежности и, как следствие -- качеству.

В этом эссе я попробую привести исторические моменты, так или иначе связанные с вопросом обеспечения качества программного обеспечения и показать их методологический аппарат. Мы увидим различные трактовки понятия качества программного обеспечения, увидим, какие существуют методы для обеспечения качества ПО и поближе познакомимся в один из них -- формальной верификацией.
