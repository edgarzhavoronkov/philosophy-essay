\subsection{Темпоральная логика}

Теоретической основой для проверки моделей является, в частности, так называемая темпоральная логика. От более или менее привычной нам классической или интуиционистской логики она отличается аппарата, позволяющего рассуждать о высказываниях во временн\'{о}м аспекте. Еще во времена Аристотеля, стало понятно, что некоторым высказываниям нельзя приписать определенное истинностное значение в текущий момент времени, например <<завтра флоты столкнутся в битве>>. Эту проблему изучали философы мегарской школы, которые, в частности, ввели подобие темпоральных операторов <<возможно>> и <<необходимо>>.

В средние века темпоральной логикой занимался Уильям Оккамский(англ. William of Ockham), который выдвинул идею о том, что человек не знает истинностное значение высказываний, относящихся к будущему по той причине, что это значение известно только лишь богу. Однако, по его мнению, человек волен выбирать между различными возможными сценариями развития будущего. Современная же темпоральная логика появилась в пятидесятых годах прошлого столетия в работах Артура Приора(англ. Arthur Prior), например в \cite{prior1958time}.

Интуитивно, темпоральную логику можно представить, как формальную систему, в которой кроме обычных логических связок(конъюнкция, дизъюнкция, отрицание и импликация) есть еще и так называемые модальные связки, которые определяются следующим образом:

Бинарные:
\begin{enumerate}
  \item \textbf{Until}. Выражение $a\ \mathcal{U}\ b$ означает, что формула $a$ истинна до того момента, когда начинает быть истинна формула $b$.
  \item \textbf{Release}. Выражение $a\ \mathcal{R}\ b$ означает, что формула $a$ <<освобождает>> формулу $b$, то есть перестает быть истинной, если $b$ истинна. Причем, это происходит, пока не наступит тот момент, когда $a$ впервые станет истинна(если этот момент не наступает, то это происходит всегда). Иначе, $a$ должна хотя бы раз стать истинной, пока $b$ не стала истинной первый раз.
\end{enumerate}

Унарные:
\begin{enumerate}
  \item \textbf{Next}. Выражение $\mathcal{N}\ a$ означает, что формула $a$ должна стать истинной в момент времени, непосредственно следующий за данным.
  \item \textbf{Future}. Выражение $\mathcal{F}\ a$ означает, что формула $a$ должна стать истинной хотя бы в один из последующих моментов времени.
  \item \textbf{Globally}. Выражение $\mathcal{G}\ a$ означает, что формула $a$ должна быть истинной во всех будущих моментах моментах времени.
  \item \textbf{All}. Выражение $\mathcal{A}\ a$ означает, что формула $a$ должна становиться истинной на всех ветвях, начинающихся с данного момента времени.
  \item \textbf{Exists}. Выражение $\mathcal{A}\ a$ означает, что существует хотя бы одна ветвь, на которой формула $a$ становится истинной.
\end{enumerate}

Как мы увидели, темпоральная логика позволяет нам формализовывать действия, которые происходят во времени. Это означает , что с ее помощью можно, например, рассуждать о свойствах программ, которые выполняют какие-либо действия на протяжении какого-либо временн\'{о}го промежутка. Недостатком же темпоральной логики может выступать сложность конструкций, которые нам приходится формулировать в ее языке для сложных программных систем.
