\subsection{Темпоральная логика}

Теоретической основой для проверки моделей является, в частности, так называемая темпоральная логика. От более или менее привычной нам классической или интуиционистской логики она отличается аппарата, позволяющего рассуждать о высказываниях во временн\'{о}м аспекте. Еще во времена Аристотеля, стало понятно, что некоторым высказываниям нельзя приписать определенное истинностное значение в текущий момент времени, например <<завтра флоты столкнутся в битве>>. Эту проблему изучали философы мегарской школы, которые, в частности, ввели подобие темпоральных операторов <<возможно>> и <<необходимо>>.

В средние века темпоральной логикой занимался Уильям Оккамский(англ. William of Ockham), который выдвинул идею о том, что человек не знает истинностное значение высказываний, относящихся к будущему по той причине, что это значение известно только лишь богу. Однако, по его мнению, человек волен выбирать между различными возможными сценариями развития будущего. Современная же темпоральная логика появилась в пятидесятых годах прошлого столетия в работах Артура Приора(англ. Arthur Prior), например в \cite{prior1958time}.

Интуитивно, темпоральную логику можно представить, как формальную систему, в которой кроме обычных логических связок(конъюнкция, дизъюнкция, отрицание и импликация) есть еще и так называемые модальные связки, которые определяются следующим образом:

Бинарные:
\begin{enumerate}
  \item \textbf{Until}
  \item \textbf{Release}
\end{enumerate}

Унарные:
\begin{enumerate}
  \item \textbf{Next}
  \item \textbf{Future}
  \item \textbf{Globally}
  \item \textbf{All}
  \item \textbf{Exists}
\end{enumerate}
