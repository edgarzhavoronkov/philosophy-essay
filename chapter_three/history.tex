\subsection{Историческая справка}

Согласно \cite{omodeo2017martin}, еще в пятидесятых годах прошлого века Мартином Девисом(англ. Martin Davis) было представлено первое формальное доказательство, полученное автоматически -- доказательство того, что произведение двух четных чисел четно в арифметике Пресбургера. Спустя короткое время, уже в конце шестидесятых, начали появляться первые автоматические доказатели теорем, которые использовались для верификации программ, написанных на таких языках, как Pascal, Ada и Java.

Еще позже, в 1972 году, Робином Милнером(англ. Robin Milner) был создана система проверки доказательств LCF(Logic for Computable Functions), которая положила начало современным системам автоматического доказательства теорем, таким как HOL или Coq. На текущий момент, Coq является одним из самых популярных инструментов формальной верификации и автоматического доказательства теорем.

У доказателей теорем есть один недостаток, который залючается в том, что если некоторое утверждение \textbf{не} является теоремой в некоторой логике, то доказатель ничего не сможет вам об этом сказать.

Для верификации конкурентных(многопоточных) программ широко используются другие методы, такие как проверка моделей или Сети Петри. Их история начинается в восьмидесятых годах с работ Аллена Эмерсона и Эдмунда Кларка(англ. E. Allen Emerson и Edmund M. Clarke) о применении в этой области темпоральной логики, например -- \cite{Clarke:1981:DSS:648063.747438} 

Если оторваться от контекста индустрии разработки ПО, то принцип верификации был выдвинут еще в тридцатых годах. Студенты и преподаватели кафедры философии индуктивных наук Венского университета собирали семинар, более известный как Венский Кружок. Именно там под влиянием идей еще Эрнста Маха(нем. Ernst Mach), выдвинувшего мысль о том, что <<суждения, которые не могут быть ни проверены, ни отвергнуты не имеют отношения к науке>> -- цитата по \cite{wiki:mach} был сформулирован принцип верификационизма, утверждавший о том, что к реальному миру имеют отношения лишь те суждения, которые можно проверить экспериментом.
