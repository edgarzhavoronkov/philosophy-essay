\subsection{Статические методы}

Под статическими методами понимают все методы, которые используют различные артефакты, полученные при проектировании или разработке. Такими артефактами, например, являются требования, спецификации или сам программный код. Важная особенность статических методов заключается в том, что они никоим образом не используют результат выполнения программы. К таки методам относят например, статический анализ кода или формальную верификацию.

Статический анализ кода это процесс выявления ошибок и недочетов в исходном коде, который можно рассматривать как автоматизированный вариант код-ревью. Помимо собственно выявления ошибок, статический анализ позволяет решать задачи рекомендаций по оформлению кода и подсчет метрик кода(например цикломатической сложности или сцепления между двумя программными модулями).

Статические анализаторы обладают тем преимуществом, что позволяют находить огромное количество ошибок еще на этапе программирования, что существенно снижает стоимость их исправления. Кроме того статический анализатор обеспечивает:

\begin{enumerate}
  \item Полное покрытие кода. Участки кода, которые редко получают управление обычно остаются без внимания в ходе код-ревью, между тем являясь потенциальным источником ошибок.
  \item Так как статический анализ не использует результат запуска программы, то он не зависит от окружения, в котором она может исполняться, тем самым позволяя находить ошибки, которые проявляются при смене окружения(компилятора, аппаратной платформы и т. д.)
  \item Не секрет, что человек часто опечатывается по невнимательности и такие ошибки очень сложно обнаружить. Статические анализаторы с легкостью находят ошибки такого рода и позволяют сэкономить огромное количество времени.
\end{enumerate}

Несмотря на кажущуюся современность, история статического анализа кода начинается примерно в 1970-ых годах, с появления утиллиты lint в операционной системе Unix, которая описана в работе Стивена Джонсона(англ. Stephen C. Johnson) \cite{Johnson78lint}. Именно ее можно считать первым статическим анализатором. С современной тоски зрения она была довольно примитивной, но именно она дала начало развитию статических анализаторов.

Формальные методы, в отличии от статического анализа кода, опираются на математический аппарат в ожидании, что его использование существенно повышает надежность разрабатываемой системы. При этом, они довольно сложны и зачастую основываются на не всегда достижимых в реальности предположениях. Формальные методы можно применять на трех уровнях:

\begin{enumerate}
  \item Нулевой. Разрабатывают формальную спецификацию, как артефакт, с оглядкой на который в дальнейшем идет разработка. Очевидно, в этом случае мы все еще не можем гарантировать, что написанный код соответствует спецификации, для чего нужен следующий уровень.
  \item Первый. Программный код \textbf{выводится} из формальной спецификации автоматически, что позволяет неформально рассуждать о том, насколько он соответствует спецификации.
  \item Второй. Полностью формализованные доказательства выводятся и проверяются автоматически.
\end{enumerate}

Формальная верификация занимает первый уровень применения формальных методов. Некоторые конкретные механизмы мы рассмотрим в следующей главе, а здесь же просто скажем, что она собой представляет. Теоретически, основой формальной верификации служит доказательство на некой абстрактной математической модели свойств программной системы в предположении о том, что эта модель адекватно отражает природу вышеозначенной системы. Зачастую для моделирования используют:

\begin{enumerate}
  \item Формальную семантику языка программирования
  \item Теории типов
  \item Абстрактные автоматы
  \item Вычислительные формализмы, такие как машина Тьюринга или Поста или $\lambda$-исчисление
  \item и т.д.
\end{enumerate}
