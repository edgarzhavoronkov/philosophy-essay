\subsection{Динамические методы}

К динамическим методам мы отнесем все методы, которые используют результат запуска программы. Основная цель таких методов -- измерить производительность программы в заданных условиях или обнаружить ошибки. Примерами таких методов являются профилирование и тестирование программного обеспечения.

Профилирование представляет собой сбор и измерение характеристик работы программы с целью последующей ее оптимизации. Характеристиками могут выступать время работы программы, используемые ресурсы компьютера, частота вызова определенных функций и так далее. Инструмент, который осуществляет профилирование называется профилировщиком(profiler).

Первым профилировщиком можно считать утиллиту prof из операционной системы Unix, которая занималась тем, что собирала данные о том, сколько времени и с какой частотой происходили вызовы функций в программах, написанных под эту операционную систему. Дальнейшие профилировщики, такие как gprof развивали эту технику, за счет построения и анализа графа вызовов(call graph) функций в программе, что позволяло оценивать, насколько часто та или иная функция вызывается относительно других и какое место она занимает в цепочке вызовов. -- \cite{Graham:1982:GCG:872726.806987}.

Современные профилировщики позволяют производить так называемое инструментирование -- модификацию кода программы, с целью сбора данных для анализа. Кроме профилирования, инструментирование полезно в контексте анализа кода в средах разработки.

Тестирование программного обеспечения же, предполагает запуск программы с целью нахождения ошибок. Важным моментом является тот факт, что тестирование именно \textbf{находит} ошибки, в отличии от формальных методов, вкратце описанных раннее, которые \textbf{доказывают} отсутствие ошибок по отношению к спецификации программы. На этот счет широко известна цитата Эдсгера Дейкстры(нидерл. Edsger Wybe Dijkstra) -- <<тестирование показывает не отсутствие ошибок, но их наличие>> из \cite{buxton1970software}.

Мы уже приводили некоторые виды тестирования программного обеспечения в первом разделе этой главы, поэтому не будем повторяться, скажем лишь, что несмотря на кажущуюся ненадежность, тестирование все еще остается одним из самых популярных методов обеспечения качества программного обеспечения, который позволяет убедиться в том, что программа на базовом уровне отвечает требованиям, предъявленным к ней в спецификации.

Мнногобразие видов тестирования позволяет получать большое количество информации для анализа, а возможность автоматизации позволяет удобно встроить тестирование в жизненный цикл разработки. Например -- автоматический запуск тестов при пересборке прорграммы, что позволяет быстро находить ошибки в программе еще на этапе разработки.
